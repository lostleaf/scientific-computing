%%%%%%%%%%%%%%%%%%%%%%%%%%%%%%%%%%%%%%%%%%%%%%%%%%%%%%%%%%%%%%%%%%%%%%%%
\documentclass[12pt]{article}
\usepackage{amssymb}
\usepackage{amsmath,amssymb,CJK}
\usepackage{graphicx}
\usepackage{subfigure}
\usepackage{listings}
\usepackage{enumerate}

\openup 7pt\pagestyle{plain} \topmargin -40pt \textwidth
14.5cm\textheight 21.5cm
\parskip .09 truein
\baselineskip 4pt\lineskip 4pt \setcounter{page}{1}
\def\a{\alpha}\def\b{\beta}\def\d{\delta}\def\D{\Delta}\def\fs{\footnotesize}
\def\g{\gamma}
\def\G{\Gamma}\def\l{\lambda}\def\L{\Lambda}\def\o{\omiga}\def\p{\psi}
\def\se{\subseteq}\def\seq{\subseteq}\def\Si{\Sigma}\def\si{\sigma}\def\vp{\varphi}\def\es{\varepsilon}
\def\sc{\scriptstyle}\def\ssc{\scriptscriptstyle}\def\dis{\displaystyle}
\def\cl{\centerline}\def\ll{\leftline}\def\rl{\rightline}\def\nl{\newline}
\def\ol{\overline}\def\ul{\underline}\def\wt{\widetilde}\def\wh{\widehat}
\def\rar{\rightarrow}\def\Rar{\Rightarrow}\def\lar{\leftarrow}
\def\Lar{\Leftarrow}\def\Rla{\rightleftarrow}\def\bs{\backslash}
\def\ra{\rangle}\def\la{\langle}\def\hs{\hspace*}\def\rb{\raisebox}
\def\ni{\noindent}\def\hi{\hangindent}\def\ha{\hangafter}
\def\vs{\vspace*}
\def\hom#1{\mbox{\rm Hom($#1,#1$)}}
\def\thebeg{\vskip8pt\par\ni}
\def\theend{\vskip5pt\par}
\def\chabeg{\pagebreak\par}
\def\chaend{\vskip20pt\par}
\def\secbeg{\vskip15pt\par}
\def\secend{\vskip10pt\par}
\def\exebeg{\vskip10pt}
\def\exeend{\vskip6pt}
\def\undot#1{\mbox{$\mbox{#1}\hs{-1.5ex}_{_{\dis\centerdot}}\,\,$}}
\def\qed{\hfill\mbox{$\Box$}}
\def\C{\mathbb{C}}
\def\Q{\mathbb{Q}}
\def\ii{\mbox{\,{\bf i}\,}}
\def\jj{\mbox{\,{\bf j}\,}}
\def\AA{{\cal A}}
\def\BB{{\cal B}}
\def\DD{{\cal D}}
\def\EE{{\mbox{\bf 1}}}
\def\OO{{\mbox{\bf 0}}}
\def\kk{{\mbox{\bf k}}}
\def\R{\mathbb{R}}
\def\F{\mathbb{F}{\ssc\,}}
%\def\similar{\rb{-4pt}{\mbox{\,\~\,}}}
\def\similar{\backsim}
\def\Llra{\Longleftrightarrow}
\def\Lra{\Longrightarrow}
\def\Lla{\Longleftarrow}
\def\mat#1#2{\mbox{$\left(\begin{array}{#1}#2\end{array}\right)$}}
\def\det#1#2{\mbox{$\left|\begin{array}{#1}#2\end{array}\right|$}}
\def\eset{\emptyset}
\parindent=5ex
\setlength{\parindent}{0pt}
\setlength{\parskip}{1ex plus 0.5ex minus 0.2ex}
\newtheorem{Example}{\text{例}}
\begin{CJK*}{UTF8}{gbsn}

\date{}
\title{Hill密码的加密、解密与破译}
\author{李青林, 5110309074\\郑辉煌, 5110209289\\}
\begin{document}
\maketitle
\section*{实验任务1}
\subsection*{问题描述}
在问题(2)中,若已知密文前4个字母OJWP分别代表TACO,问能否将此段密码破译?
\subsection*{解答}
密文:
$\begin{pmatrix}
\text{O}\\ \text{J}
\end{pmatrix}\longrightarrow$
明文:
$\begin{pmatrix}
\text{T}\\ \text{A}
\end{pmatrix}$,
密文:
$\begin{pmatrix}
\text{W}\\ \text{P}
\end{pmatrix}\longrightarrow$
明文:
$\begin{pmatrix}
\text{C}\\ \text{O}
\end{pmatrix}$\\

$\begin{pmatrix}
\text{O}\\ \text{J}
\end{pmatrix}\leftrightarrow\beta_1=
\begin{pmatrix}
15\\10
\end{pmatrix}=A\alpha_1\Leftrightarrow\alpha_1=
\begin{pmatrix}
20\\1
\end{pmatrix}\leftrightarrow
\begin{pmatrix}
\text{T}\\ \text{A}
\end{pmatrix}
$\\
$\begin{pmatrix}
\text{W}\\ \text{P}
\end{pmatrix}\leftrightarrow\beta_1=
\begin{pmatrix}
23\\16
\end{pmatrix}=A\alpha_1\Leftrightarrow\alpha_1=
\begin{pmatrix}
3\\15
\end{pmatrix}\leftrightarrow
\begin{pmatrix}
\text{C}\\ \text{O}
\end{pmatrix}$\\

$\textrm{det}(\beta_1,\beta_2)=
\begin{vmatrix}
15 & 23\\
10 & 16\\
\end{vmatrix}=10
$\\
$\textrm{gcd}(10,26)=2 \Rightarrow \beta_1,\beta_2\text{在模26下线性相关}$\\
因此无法解密\\
\newpage
\section*{实验任务2}
\subsection*{问题描述}
设英文$26$个字母以下面乱序表与$Z_{26}$中的整数对应:

\begin{tabular}{|c|c|c|c|c|c|c|c|c|c|c|c|c|}
\hline
A & B & C & D & E & F & G & H & I & J & K & L & M \\
\hline
5 & 23 & 2 & 20 & 10 & 15 & 8 & 4 & 18 & 25 & 0 & 16 & 13 \\
\hline
N & O & P & Q & R & S & T & U & V & W & X & Y & Z \\
\hline
7 & 3 & 1 & 19 & 6 & 12 & 24 & 21 & 17 & 14 & 22 & 11 & 9 \\
\hline
\end{tabular}

\begin{enumerate}
\item
设$A=
\begin{pmatrix}
8 & 6 & 9 & 5\\
6 & 9 & 5 & 10\\
5 & 8 & 4 & 9\\
10 & 6 & 11 & 4\\
\end{pmatrix}
$, 验证矩阵$A$能否作为$\text{Hill}_4$密码体制的加密矩阵。
\item
设明文为\\
HILL CRYPTOGRAPHIC SYSTEM IS TRADITIONAL\\
利用上面的表值与加密矩阵给此明文加密,并将得到的密文解密.
\item
已知在上述给定表值下的一段$\text{Hill}_4$密码的密文为\\
JCOW ZLVB DVLE QMXC\\
对应的明文为\\
DELAY OPERATIONSU\\
能否确定加密矩阵?
\end{enumerate}

\subsection*{解答}
\begin{enumerate}
\item
$\textrm{det}(A)=25\quad(\mathrm{mod}~26)$\\
$\mathrm{gcd}(25, 26)=1\Longrightarrow A$在模26下可逆\\
因此$A$可以作为加密矩阵\\

\item
对明文分组:\\
HILL CRYP TOGR APHI CSYS TEMI STRA DITI ONAL\\
构造$4$维向量\\
$\begin{pmatrix}18\\25\\13\\13\end{pmatrix}$,
$\begin{pmatrix}20\\12\\9\\19\end{pmatrix}$,
$\begin{pmatrix}21\\1\\4\\12\end{pmatrix}$,
$\begin{pmatrix}23\\19\\18\\25\end{pmatrix}$,
$\begin{pmatrix}20\\24\\9\\24\end{pmatrix}$,
$\begin{pmatrix}21\\15\\7\\25\end{pmatrix}$,
$\begin{pmatrix}24\\21\\12\\23\end{pmatrix}$,
$\begin{pmatrix}10\\25\\21\\25\end{pmatrix}$,
$\begin{pmatrix}1\\3\\23\\13\end{pmatrix}$\\
用$A$左乘得\\
$\begin{pmatrix}8\\8\\17\\5\end{pmatrix}$,
$\begin{pmatrix}18\\21\\13\\5\end{pmatrix}$,
$\begin{pmatrix}10\\15\\3\\22\end{pmatrix}$,
$\begin{pmatrix}13\\25\\18\\18\end{pmatrix}$,
$\begin{pmatrix}11\\23\\24\\19\end{pmatrix}$,
$\begin{pmatrix}4\\0\\10\\9\end{pmatrix}$,
$\begin{pmatrix}21\\25\\23\\18\end{pmatrix}$,
$\begin{pmatrix}24\\16\\13\\9\end{pmatrix}$,
$\begin{pmatrix}12\\18\\4\\21\end{pmatrix}$\\
查表得密文为:\\
IJMM DSZQ UPHS BQIJ DTZT UFNJ TUSB EJUJ POBM\\

$A^{-1}=
\begin{pmatrix}
23&   20&    5&    1&\\
2&   11&   18&    1&\\
2&   20&    6&   25&\\
25&    2&   22&   25&\\
\end{pmatrix}
$\\
用$A^{-1}$左乘密文向量得\\
$\begin{pmatrix}18\\25\\13\\13\end{pmatrix}$,
$\begin{pmatrix}20\\12\\9\\19\end{pmatrix}$,
$\begin{pmatrix}21\\1\\4\\12\end{pmatrix}$,
$\begin{pmatrix}23\\19\\18\\25\end{pmatrix}$,
$\begin{pmatrix}20\\24\\9\\24\end{pmatrix}$,
$\begin{pmatrix}21\\15\\7\\25\end{pmatrix}$,
$\begin{pmatrix}24\\21\\12\\23\end{pmatrix}$,
$\begin{pmatrix}10\\25\\21\\25\end{pmatrix}$,
$\begin{pmatrix}1\\3\\23\\13\end{pmatrix}$\\
查表即可得明文

\item
对明文分组:\\
DELA YOPE RATI ONSU\\
明文向量:\\
$\alpha_1=\begin{pmatrix}20\\10\\16\\5\end{pmatrix}$,
$\alpha_2=\begin{pmatrix}11\\3\\1\\10\end{pmatrix}$,
$\alpha_3=\begin{pmatrix}6\\5\\24\\18\end{pmatrix}$,
$\alpha_4=\begin{pmatrix}3\\7\\12\\21\end{pmatrix}$\\
$\mathrm{det}(\alpha_1,\alpha_2,\alpha_3,\alpha_4)=
\begin{vmatrix}
   25 &   9   &20&   19\\
    2  & 16  & 17 &  13\\
    3   &17 &  16  & 22\\
   14   &23&   10   & 2\\
\end{vmatrix}
=15\quad(\mathrm{mod}~26)
$\\
$\mathrm{gcd}(26,15)=1\Rightarrow\alpha_1,\alpha_2,\alpha_3,\alpha_4$在模26下线性无关\\
密文向量:\\
$\beta_1=\begin{pmatrix}25\\2\\3\\14\end{pmatrix}$,
$\beta_2=\begin{pmatrix}9\\16\\17\\23\end{pmatrix}$,
$\beta_3=\begin{pmatrix}20\\17\\16\\10\end{pmatrix}$,
$\beta_4=\begin{pmatrix}19\\13\\22\\2\end{pmatrix}$\\
$\mathrm{det}(\beta_1,\beta_2,\beta_3,\beta_4)=
\begin{vmatrix}
   25 &   9   &20&   19\\
    2  & 16  & 17 &  13\\
    3   &17 &  16  & 22\\
   14   &23&   10   & 2\\
\end{vmatrix}
=11\quad(\mathrm{mod}~26)
$\\
$\mathrm{gcd}(26,11)=1\Rightarrow\beta_1,\beta_2,\beta_3,\beta_4$在模26下线性无关\\
设加密矩阵为$A$,则有$A(\alpha_1,\alpha_2,\alpha_3,\alpha_4)=(\beta_1,\beta_2,\beta_3,\beta_4)$\\
设$C=(\alpha_1,\alpha_2,\alpha_3,\alpha_4), P=(\beta_1,\beta_2,\beta_3,\beta_4)$\\
$A=PC^{-1}=
\begin{pmatrix}
    8&    6   & 9&    5\\
    6 &   9  &  5 &  10\\
    5  &  8 &   4  &  9\\
   10   & 6&   11   & 4\\

\end{pmatrix}
$

\end{enumerate}
\section*{实验任务3}
    \subsection*{问题描述}
    设已知一份密文为Hill$_{2}$密码体系,其中出现频数最高的双字母是RH和NI,而在明文语言中,出现频数最高的双字母为TH和HE,由这些信息按下表给出的表值能得到什么样的加密矩阵?
    \begin{center}
      \begin{tabular}{|c|c|c|c|c|c|c|c|c|c|c|c|c|}
        \hline
        % after \\: \hline or \cline{col1-col2} \cline{col3-col4} ...
        A & B & C & D & E & F & G & H & I & J & K & L & M \\ \hline
        0 & 1 & 2 & 3 & 4 & 5 & 6 & 7 & 8 & 9 & 10 & 11 & 12 \\ \hline \hline
        N & O & P & Q & R & S & T & U & V & W & X & Y & Z \\ \hline
        13 & 14 & 15 & 16 & 17 & 18 & 19 & 20 & 21 & 22 & 23 & 24 & 25 \\ \hline
      \end{tabular}
    \end{center}
    \subsection*{解答}
    $$
    \begin{pmatrix}
        R\\
        H\\  
    \end{pmatrix}
    \leftrightarrow 
    \begin{pmatrix}
        17\\
        7\\
    \end{pmatrix}
    ,
    \begin{pmatrix}
        N\\
        I\\
    \end{pmatrix}
    \leftrightarrow
    \begin{pmatrix}
        13\\
        8\\
    \end{pmatrix}
    $$
    $$
    \begin{pmatrix}
        T\\
        H\\
    \end{pmatrix}
    \leftrightarrow
    \begin{pmatrix}
        19\\
        7\\
    \end{pmatrix}
    ,
    \begin{pmatrix}
        H\\
        E\\
    \end{pmatrix}
    \leftrightarrow
    \begin{pmatrix}
        7\\
        4\\
    \end{pmatrix}
    $$
    记
    $$ P = 
        \begin{pmatrix}
            17 & 13\\
            7 & 8\\
        \end{pmatrix},
      C = 
        \begin{pmatrix}
            19 & 7\\
            7 & 4\\
        \end{pmatrix}
    $$
    设加密矩阵为$A$,则$P = AC(mod\ 26)$,所以$A = PC^{-1}(mod\ 26)$\\
    通过matlab代码:
    \begin{lstlisting}[language=MATLAB]
    function Y = invmod( P, C )
        %mod26 inverse matrix
        %for more detail to see <<math experiments>> in page 109
        %D = det(P);
        D = P(2, 2) * P(1, 1) - P(1, 2) * P (2, 1);
        if gcd(D, 26) ~= 1;
            disp('Error');
        else
            for i = 1: 25
                if mod(i * D, 26) == 1
                    break;
                end
            end
            invD = i;
            Q(1, 1) =  P(2, 2);
            Q(1, 2) = -P(1, 2);
            Q(2, 1) = -P(2, 1);
            Q(2, 2) = P(1, 1);
            Y = mod(Q * invD, 26);
        end
        Y = mod(C * Y, 26);
    end
    \end{lstlisting}
    上面针对课本代码的改进是防止了det(P)出现不是整数和inv(P)会有计算机数据误差情况。再通过调用:\\
    \begin{lstlisting}[language=MATLAB]
    >> P = [17, 13; 7, 8];
    >> C = [19. 7; 7, 4];
    >> A = invmod(C, P)
    \end{lstlisting}
    得到加密矩阵
    $$ A = 
        \begin{pmatrix}
            3 & 24\\
            24 & 25\\
        \end{pmatrix}
    $$
\section*{实验任务4}
    \subsection*{问题描述}
        如下的密文根据课本表10.1以Hill$_{2}$加密,密文为\\
        VIKYNOTCLKYRJQETIRECVUZLNOJTUYDIMHRCFITQ\\
        已获知其中相邻字母LK代表字母KE,试破译这份密文。\\
    \subsection*{解答}
        $$
        \begin{pmatrix}
            L\\
            K\\
        \end{pmatrix}    
        =
        \begin{pmatrix}
            12\\
            11\\
        \end{pmatrix}
        $$
        $$
        \begin{pmatrix}
            K\\
            E\\
        \end{pmatrix}
        =
        \begin{pmatrix}
            11\\
            5\\
        \end{pmatrix}
        $$
        设解密矩阵为$B = \begin{pmatrix} x_{1} & x_{2} \\ x_{3} & x_{4} \\\end{pmatrix}$,则\\
        $$
        \begin{pmatrix}
            x_{1} & x_{2} \\
            x_{3} & x_{4} \\
        \end{pmatrix}
        \begin{pmatrix}
            12 \\
            11 \\
        \end{pmatrix}
        =
        \begin{pmatrix}
            11 \\
            5 \\
        \end{pmatrix}
        (mod \ 26)
        $$
        从而解得通解为
        $$
        \begin{pmatrix}
            x_{1} & x_{2} \\
            x_{3} & x_{4} \\
        \end{pmatrix}
        =
        \begin{pmatrix}
            c_{1} & 1+6c_{1} \\
            c_{2} & 17+6c_{2} \\
        \end{pmatrix} 
        $$
        一开始以C++代码枚举所有可能的$c_{1}$和$c_{2}$的值,发现所有的密文第5,6个字符“NO”对应的明文都是“OU”,之后猜测明文中KE可能是汉语拼音的“可”,故看后面两个明文是不“YI”对应汉语的“以”,失败了,再转而猜测是英语,猜KE可能是MAKE的后两个字母,用C++代码验证:\\
        \begin{lstlisting}[language=C++]
#include <iostream>
#include <cstdio>
#include <cstdlib>

using namespace std;

const int MAXN = 200;
char code[MAXN] = "VIKYNOTCLKYRJQETIRECVUZLNOJTUYDIMHRCFITQ";
char ans[MAXN];
int length;
int count = 0;
int getNum(char letter)
{
    return (letter - 'A' + 1) % 26;
}

char getLetter(int num)
{
    if(num == 0) return 'Z';
    return 'A' + num - 1;
}

void printCode(int c1, int c2)
{
    int a1, a2, b1, b2;
    int d1 = (1 + 6 * c1) % 26;
    int d2 = (17 + 6 * c2) % 26;
    for(int i = 0; i < length; i += 2)
    {
        b1 = getNum(code[i]);
        b2 = getNum(code[i + 1]);

        a1 = (c1 * b1 + d1 * b2) % 26;
        a2 = (c2 * b1 + d2 * b2) % 26;

        ans[i] = getLetter(a1);
        ans[i + 1] = getLetter(a2);
        if(code[i] == 'T' && code[i+1] == 'C')
        {
            if(ans[i] != 'M' || ans[i+1] != 'A') return;
            //to see the out file, I can see ans[4] and ans[5] 
            //must be 'O' and 'U', so
            //guess ans[3~9] is "you make",
            // so try it and final success!!
        }
        //printf("%c%c", getLetter(a1), getLetter(a2));
    }
    ans[length] = '\0';
    ++count;
    printf("%d\n", count);
    printf(ans);
    printf("\n");
}
int main()
{
    freopen("in.txt", "w", stdout);
    for(length = 0; code[length] != '\0'; ++length);

    for(int i = 0; i < 26; ++i)
        for(int j = 0; j < 26; ++j)
            printCode(i, j);

    return 0;
}
        \end{lstlisting}
        得到的输出结果为:\\
        1\\
        CANLOUMAKEAAOMEYEGTRWITHOUTBRRAKIAGEGGSS\\
        2\\
        CANYOUMAKEANOMELETTEWITHOUTBREAKINGEGGSS\\
        3\\
        CAALOUMAKENAOMRYRGGRWITHOUTBERAKVAGEGGSS\\
        4\\
        CAAYOUMAKENNOMRLRTGEWITHOUTBEEAKVNGEGGSS\\
        注意到第2条,可认为是明文:Can you make an omelette without breaking eggs (最后一个s为哑字母)\\
\section*{实验任务5}
\subsection*{问题描述}
若截获一下密文\\
CKYNOHKQMAXJQBHAZWUHDAOQWXIPQZBKMPUTIPVSWSBYXKKWQHADMBDM\\
已知它是根据Hill$_2$体制加密的,能否将它解密?
\subsection*{解答法1}
    基于字母频数统计的方法:\\
    查阅资料得汉语拼音的字母出现频率(\%)头几名为:\\
    I(12.93), N(12.56), G(9.50), U(9.40), A(8.22)\\
    英语出现频率高的为:\\
    E(12.95), T(9.41), A(8.19), O(7.26), N(7.06)\\

    统计得密文共有56个字符,故若明文是拼音,则字母出现频数应满足:
    $$ I \approx 56 * 12.93 \% \approx 7.2 $$
    $$ N \approx 56 * 12.56 \% \approx 7.0 $$
    $$ G \approx 56 * 9.5 \% \approx 5.3 $$
    $$ U \approx 56 * 9.4 \% \approx 5.3 $$
    $$ A \approx 56 * 8.22 \% \approx 4.6 $$
    考虑误差,现在将密文用所有可能的Hill$_{2}$解密矩阵翻译成明文,首先,明文第一个拼音中出现A,E,I,O,U的概率极小,将这部分数据删去,再将明文中I出现次数小于6个,N小于5 个,G、U、A 小于4 个的数据删去。幸运的是,在这时就得到了答案。(若此时没得到答案,则明文有可能是英语,那么对英文出现频率高的字母同理筛选)
    下面给出C++代码:
    \begin{lstlisting}[language=C++]
#include <iostream>
#include <cstdio>
#include <cstdlib>

using namespace std;

const int MAXN = 200;
char code[MAXN] = "CKYNOHKQMAXJQBHAZWUHDAOQWXIP
QZBKMPUTIPVSWSBYXKKWQHADMBDM";
char ans[MAXN];
int length;
int count = 0;
int getNum(char letter)
{
    return (letter - 'A' + 1) % 26;
}

char getLetter(int num)
{
    if(num == 0) return 'Z';
    return 'A' + num - 1;
}

int gcd(int x, int y)
{
    if(x == 0) return y;
    return gcd(y % x, x);
}
bool analysis()
{
    int count_i = 0,count_n = 0,count_g = 0,count_a = 0,count_u = 0;
    if(ans[0] == 'A' || ans[0] == 'E' || ans[0] == 'I' 
       || ans[0] == 'O' || ans[0] == 'U') return false;
    for(int i = 0; i < length; ++i)
    {
        if(ans[i] == 'A') ++count_a;
        else if(ans[i] == 'U') ++count_u;
        else if(ans[i] == 'I') ++count_i;
        else if(ans[i] == 'N') ++count_n;
        else if(ans[i] == 'G') ++count_g;
    }

    if(count_i < 6) return false;
    if(count_n < 5) return false;
    if(count_g < 4) return false;
    if(count_a < 4) return false;
    if(count_u < 4) return false;
    return true;
}

void printCode(int a1, int a2, int a3, int a4)
{
    int det = a1 * a4 - a2 * a3;
    if(det == 0) return;
    if(det > 0 && gcd(det, 26) != 1) return;

    int b1, b2, o1, o2;
    for(int i = 0; i < length; i += 2)
    {
        b1 = getNum(code[i]);
        b2 = getNum(code[i + 1]);

        o1 = (a1 * b1 + a2 * b2) % 26;
        o2 = (a3 * b1 + a4 * b2) % 26;

        ans[i] = getLetter(o1);
        ans[i + 1] = getLetter(o2);


    }
    ans[length] = '\0';
    if(analysis())
    {
        ++count;
        printf("%d\n", count);
        printf(ans);
        printf("\n");
    }
}
int main()
{
    freopen("in.txt", "w", stdout);
    for(length = 0; code[length] != '\0'; ++length);

    for(int i1 = 0; i1 < 26; ++i1)
        for(int i2 = 0; i2 < 26; ++i2)
            for(int i3 = 0; i3 < 26; ++i3)
                for(int i4 = 0; i4 < 26; ++i4)
                    printCode(i1, i2, i3, i4);

    return 0;
}

    \end{lstlisting}
    得到的结果为:\\
    1\\
    ZAIBENTENGZHIHOUWEIRUANYIJINGTUICHUXINYIDAICAOZUOXITONGG\\
    2\\
    KAGBGNMESGJHUHXUHEARBAIYQJINITLIEHUXINNIMARCPOWUEXWTYNVG\\
    注意到第1个答案很符合拼音用法“zai ben teng zhi hou wei ruan yi jing tui chu xin yi dai cao zuo xi tong (哑字母g)”\\
    翻译为汉语:“在奔腾之后微软已经推出新一代操作系统”\\
\subsection*{解答法2}
由于不知道任何加密信息,因此需要枚举所有可能的加密矩阵。\\
但加密矩阵总数量级达到$26^4$即上亿级别,人工识别不现实,因此采用计算机过滤+人工识别方法\\
具体做法如下:
\begin{enumerate}
\item 枚举所有合法的解密矩阵
\item 使用这些合法的矩阵对字符串解密,并使用动态规划算法作字符串匹配(字典来自网络)
\item 对匹配位数高的字符串人工识别
\end{enumerate}
解密后的字符串为\\
zaibentengzhihouweiruanyijingtuichuxinyidaicaozuoxitongg\\
对应汉语``在奔腾之后,微软已经推出新一代操作系统''\\
C++代码:
\begin{lstlisting}[language=C++]
#include <cstdio>
#include <cstdlib>
#include <cmath>
#include <cstring>
#include <string>
#include <algorithm>
using namespace std;
const int DICT_SIZE = 1050;
const int STR_LEN = 60;
char dict[DICT_SIZE][30],s[STR_LEN];
int f[STR_LEN],length[DICT_SIZE];
char *str = "CKYNOHKQMAXJQBHAZWUHDAOQWXIPQZBKMPUTIPVSWSBYXKKWQHADMBDM";
int gcd(int a,int b)
{
    return b ? gcd(b,a%b) : a;
}
int match(char *a, char *b)
{
    for(;*a && *b; a++, b++)
        if(*a != *b)
            return 0;
    return !*b;
}
void check(int a, int b, int c, int d)
{
    int det = ((a*d-b*c)%26+26)%26;
    if(gcd(det,26) > 1) return;
    memset(f,0,sizeof(f));
    int len = strlen(str);
    for(int i = 0; i < len; i += 2)
    {
        s[i]   = (a * (str[i] - 64) + b * (str[i+1] - 64)) % 26 + 96;
        s[i+1] = (c * (str[i] - 64) + d * (str[i+1] - 64)) % 26 + 96;
        if(s[i]  ==96) s[i] = 'z';
        if(s[i+1]==96) s[i+1]  = 'z';
    }
    for(int i = 0; i < len; i++)
    {
        f[i] = f[i-1];
        for(int j = 0; j <= 1000; j++)
            if (i+1 >= length[j] && match(s+i-length[j], dict[j]))
                f[i] = max(f[i], (i-length[j]>0 ? f[i-length[j]] : 0)
                +length[j]);
    }
    if(f[len-1]>30)
        printf("%s %d\n", s, f[len-1]);
}
int main()
{
    FILE *fdict = fopen("dict1.txt", "r");
    for (int i = 0; i < 1000; ++i)
    {
        fscanf(fdict, "%s", dict[i]);
        length[i] = strlen(dict[i]);
    }

    for (int i = 0; i < 25; ++i)
        for (int j = 0; j < 25 ; ++j)
            for (int k = 0; k < 25; ++k)
                for (int l = 0; l < 25; ++l)
                    check(i,j,k,l);
    return 0;
}
\end{lstlisting}
\section*{任务分工}
    任务1,任务2,任务5解法2,李青林\\
    任务3,任务4,任务5解法1,郑辉煌\\
\end{CJK*}
\end{document}