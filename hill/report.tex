%%%%%%%%%%%%%%%%%%%%%%%%%%%%%%%%%%%%%%%%%%%%%%%%%%%%%%%%%%%%%%%%%%%%%%%%
\documentclass[12pt]{article}
\usepackage{amssymb}
\usepackage{amsmath,amssymb,CJK}
\usepackage{graphicx}
\usepackage{subfigure}
\usepackage{listings}
\usepackage{enumerate}

\openup 7pt\pagestyle{plain} \topmargin -40pt \textwidth
14.5cm\textheight 21.5cm
\parskip .09 truein
\baselineskip 4pt\lineskip 4pt \setcounter{page}{1}
\def\a{\alpha}\def\b{\beta}\def\d{\delta}\def\D{\Delta}\def\fs{\footnotesize}
\def\g{\gamma}
\def\G{\Gamma}\def\l{\lambda}\def\L{\Lambda}\def\o{\omiga}\def\p{\psi}
\def\se{\subseteq}\def\seq{\subseteq}\def\Si{\Sigma}\def\si{\sigma}\def\vp{\varphi}\def\es{\varepsilon}
\def\sc{\scriptstyle}\def\ssc{\scriptscriptstyle}\def\dis{\displaystyle}
\def\cl{\centerline}\def\ll{\leftline}\def\rl{\rightline}\def\nl{\newline}
\def\ol{\overline}\def\ul{\underline}\def\wt{\widetilde}\def\wh{\widehat}
\def\rar{\rightarrow}\def\Rar{\Rightarrow}\def\lar{\leftarrow}
\def\Lar{\Leftarrow}\def\Rla{\rightleftarrow}\def\bs{\backslash}
\def\ra{\rangle}\def\la{\langle}\def\hs{\hspace*}\def\rb{\raisebox}
\def\ni{\noindent}\def\hi{\hangindent}\def\ha{\hangafter}
\def\vs{\vspace*}
\def\hom#1{\mbox{\rm Hom($#1,#1$)}}
\def\thebeg{\vskip8pt\par\ni}
\def\theend{\vskip5pt\par}
\def\chabeg{\pagebreak\par}
\def\chaend{\vskip20pt\par}
\def\secbeg{\vskip15pt\par}
\def\secend{\vskip10pt\par}
\def\exebeg{\vskip10pt}
\def\exeend{\vskip6pt}
\def\undot#1{\mbox{$\mbox{#1}\hs{-1.5ex}_{_{\dis\centerdot}}\,\,$}}
\def\qed{\hfill\mbox{$\Box$}}
\def\C{\mathbb{C}}
\def\Q{\mathbb{Q}}
\def\ii{\mbox{\,{\bf i}\,}}
\def\jj{\mbox{\,{\bf j}\,}}
\def\AA{{\cal A}}
\def\BB{{\cal B}}
\def\DD{{\cal D}}
\def\EE{{\mbox{\bf 1}}}
\def\OO{{\mbox{\bf 0}}}
\def\kk{{\mbox{\bf k}}}
\def\R{\mathbb{R}}
\def\F{\mathbb{F}{\ssc\,}}
%\def\similar{\rb{-4pt}{\mbox{\,\~\,}}}
\def\similar{\backsim}
\def\Llra{\Longleftrightarrow}
\def\Lra{\Longrightarrow}
\def\Lla{\Longleftarrow}
\def\mat#1#2{\mbox{$\left(\begin{array}{#1}#2\end{array}\right)$}}
\def\det#1#2{\mbox{$\left|\begin{array}{#1}#2\end{array}\right|$}}
\def\eset{\emptyset}
\parindent=5ex
\setlength{\parindent}{0pt}
\setlength{\parskip}{1ex plus 0.5ex minus 0.2ex}
\newtheorem{Example}{\text{例}}
\begin{CJK*}{UTF8}{gbsn}

\date{}
\title{Hill密码的加密、解密与破译}
\author{李青林, 5110309074\\郑辉煌, 5110209289\\}
\begin{document}
\maketitle
\section*{实验任务1}
\subsection*{问题描述}
在问题(2)中,若已知密文前4个字母OJWP分别代表TACO,问能否将此段密码破译?
\subsection*{解答}
密文:
$\begin{pmatrix}
\text{O}\\ \text{J}
\end{pmatrix}\longrightarrow$
明文:
$\begin{pmatrix}
\text{T}\\ \text{A}
\end{pmatrix}$, 
密文:
$\begin{pmatrix}
\text{W}\\ \text{P}
\end{pmatrix}\longrightarrow$
明文:
$\begin{pmatrix}
\text{C}\\ \text{O}
\end{pmatrix}$\\

$\begin{pmatrix}
\text{O}\\ \text{J}
\end{pmatrix}\leftrightarrow\beta_1=
\begin{pmatrix}
15\\10
\end{pmatrix}=A\alpha_1\Leftrightarrow\alpha_1=
\begin{pmatrix}
20\\1
\end{pmatrix}\leftrightarrow
\begin{pmatrix}
\text{T}\\ \text{A}
\end{pmatrix}
$\\
$\begin{pmatrix}
\text{W}\\ \text{P}
\end{pmatrix}\leftrightarrow\beta_1=
\begin{pmatrix}
23\\16
\end{pmatrix}=A\alpha_1\Leftrightarrow\alpha_1=
\begin{pmatrix}
3\\15
\end{pmatrix}\leftrightarrow
\begin{pmatrix}
\text{C}\\ \text{O}
\end{pmatrix}$\\

$\textrm{det}(\beta_1,\beta_2)=
\begin{vmatrix}
15 & 23\\
10 & 16\\
\end{vmatrix}=10
$\\
$\textrm{gcd}(10,26)=2 \Rightarrow \beta_1,\beta_2\text{在模26下线性相关}$\\
因此无法解密\\
\newpage
\section*{实验任务2}
\subsection*{问题描述}
设英文$26$个字母以下面乱序表与$Z_{26}$中的整数对应:

\begin{tabular}{|c|c|c|c|c|c|c|c|c|c|c|c|c|}
\hline 
A & B & C & D & E & F & G & H & I & J & K & L & M \\ 
\hline 
5 & 23 & 2 & 20 & 10 & 15 & 8 & 4 & 18 & 25 & 0 & 16 & 13 \\ 
\hline 
N & O & P & Q & R & S & T & U & V & W & X & Y & Z \\ 
\hline 
7 & 3 & 1 & 19 & 6 & 12 & 24 & 21 & 17 & 14 & 22 & 11 & 9 \\ 
\hline 
\end{tabular} 

\begin{enumerate}
\item
设$A=
\begin{pmatrix}
8 & 6 & 9 & 5\\
6 & 9 & 5 & 10\\
5 & 8 & 4 & 9\\
10 & 6 & 11 & 4\\
\end{pmatrix}
$, 验证矩阵$A$能否作为$\text{Hill}_4$密码体制的加密矩阵。
\item
设明文为\\
HILL CRYPTOGRAPHIC SYSTEM IS TRADITIONAL\\
利用上面的表值与加密矩阵给此明文加密,并将得到的密文解密.
\item
已知在上述给定表值下的一段$\text{Hill}_4$密码的密文为\\
JCOW ZLVB DVLE QMXC\\
对应的明文为\\
DELAY OPERATIONSU\\
能否确定加密矩阵?
\end{enumerate}

\subsection*{解答}
\begin{enumerate}
\item 
$\textrm{det}(A)=25\quad(\mathrm{mod}~26)$\\
$\mathrm{gcd}(25, 26)=1\Longrightarrow A$在模26下可逆\\
因此$A$可以作为加密矩阵\\

\item
对明文分组:\\
HILL CRYP TOGR APHI CSYS TEMI STRA DITI ONAL\\
构造$4$维向量\\
$\begin{pmatrix}18\\25\\13\\13\end{pmatrix}$,
$\begin{pmatrix}20\\12\\9\\19\end{pmatrix}$,
$\begin{pmatrix}21\\1\\4\\12\end{pmatrix}$,
$\begin{pmatrix}23\\19\\18\\25\end{pmatrix}$,
$\begin{pmatrix}20\\24\\9\\24\end{pmatrix}$,
$\begin{pmatrix}21\\15\\7\\25\end{pmatrix}$,
$\begin{pmatrix}24\\21\\12\\23\end{pmatrix}$,
$\begin{pmatrix}10\\25\\21\\25\end{pmatrix}$,
$\begin{pmatrix}1\\3\\23\\13\end{pmatrix}$\\
用$A$左乘得\\
$\begin{pmatrix}8\\8\\17\\5\end{pmatrix}$,
$\begin{pmatrix}18\\21\\13\\5\end{pmatrix}$,
$\begin{pmatrix}10\\15\\3\\22\end{pmatrix}$,
$\begin{pmatrix}13\\25\\18\\18\end{pmatrix}$,
$\begin{pmatrix}11\\23\\24\\19\end{pmatrix}$,
$\begin{pmatrix}4\\0\\10\\9\end{pmatrix}$,
$\begin{pmatrix}21\\25\\23\\18\end{pmatrix}$,
$\begin{pmatrix}24\\16\\13\\9\end{pmatrix}$,
$\begin{pmatrix}12\\18\\4\\21\end{pmatrix}$\\
查表得密文为:\\
IJMM DSZQ UPHS BQIJ DTZT UFNJ TUSB EJUJ POBM\\

$A^{-1}=
\begin{pmatrix}
23&   20&    5&    1&\\
2&   11&   18&    1&\\
2&   20&    6&   25&\\
25&    2&   22&   25&\\
\end{pmatrix}
$\\
用$A^{-1}$左乘密文向量得\\
$\begin{pmatrix}18\\25\\13\\13\end{pmatrix}$,
$\begin{pmatrix}20\\12\\9\\19\end{pmatrix}$,
$\begin{pmatrix}21\\1\\4\\12\end{pmatrix}$,
$\begin{pmatrix}23\\19\\18\\25\end{pmatrix}$,
$\begin{pmatrix}20\\24\\9\\24\end{pmatrix}$,
$\begin{pmatrix}21\\15\\7\\25\end{pmatrix}$,
$\begin{pmatrix}24\\21\\12\\23\end{pmatrix}$,
$\begin{pmatrix}10\\25\\21\\25\end{pmatrix}$,
$\begin{pmatrix}1\\3\\23\\13\end{pmatrix}$\\
查表即可得明文

\item
对明文分组:\\
DELA YOPE RATI ONSU\\
明文向量:\\
$\alpha_1=\begin{pmatrix}20\\10\\16\\5\end{pmatrix}$,
$\alpha_2=\begin{pmatrix}11\\3\\1\\10\end{pmatrix}$,
$\alpha_3=\begin{pmatrix}6\\5\\24\\18\end{pmatrix}$,
$\alpha_4=\begin{pmatrix}3\\7\\12\\21\end{pmatrix}$\\
$\mathrm{det}(\alpha_1,\alpha_2,\alpha_3,\alpha_4)=
\begin{vmatrix}
   25 &   9   &20&   19\\
    2  & 16  & 17 &  13\\
    3   &17 &  16  & 22\\
   14   &23&   10   & 2\\
\end{vmatrix}
=15\quad(\mathrm{mod}~26)
$\\
$\mathrm{gcd}(26,15)=1\Rightarrow\alpha_1,\alpha_2,\alpha_3,\alpha_4$在模26下线性无关\\
密文向量:\\
$\beta_1=\begin{pmatrix}25\\2\\3\\14\end{pmatrix}$,
$\beta_2=\begin{pmatrix}9\\16\\17\\23\end{pmatrix}$,
$\beta_3=\begin{pmatrix}20\\17\\16\\10\end{pmatrix}$,
$\beta_4=\begin{pmatrix}19\\13\\22\\2\end{pmatrix}$\\
$\mathrm{det}(\beta_1,\beta_2,\beta_3,\beta_4)=
\begin{vmatrix}
   25 &   9   &20&   19\\
    2  & 16  & 17 &  13\\
    3   &17 &  16  & 22\\
   14   &23&   10   & 2\\
\end{vmatrix}
=11\quad(\mathrm{mod}~26)
$\\
$\mathrm{gcd}(26,11)=1\Rightarrow\beta_1,\beta_2,\beta_3,\beta_4$在模26下线性无关\\
设加密矩阵为$A$,则有$A(\alpha_1,\alpha_2,\alpha_3,\alpha_4)=(\beta_1,\beta_2,\beta_3,\beta_4)$\\
设$C=(\alpha_1,\alpha_2,\alpha_3,\alpha_4), P=(\beta_1,\beta_2,\beta_3,\beta_4)$\\
$A=PC^{-1}=
\begin{pmatrix}
    8&    6   & 9&    5\\
    6 &   9  &  5 &  10\\
    5  &  8 &   4  &  9\\
   10   & 6&   11   & 4\\

\end{pmatrix}
$

\end{enumerate}
	

\end{CJK*}
\end{document}