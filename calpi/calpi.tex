%%%%%%%%%%%%%%%%%%%%%%%%%%%%%%%%%%%%%%%%%%%%%%%%%%%%%%%%%%%%%%%%%%%%%%%%
\documentclass[12pt]{article}
\usepackage{amssymb}
\usepackage{amsmath,amssymb,CJK}
\usepackage{graphicx}
\usepackage{subfigure}
\usepackage{listings}

\openup 7pt\pagestyle{plain} \topmargin -40pt  \textwidth
14.5cm\textheight 21.5cm
\parskip .09 truein
\baselineskip 4pt\lineskip 4pt \setcounter{page}{1}
\def\a{\alpha}\def\b{\beta}\def\d{\delta}\def\D{\Delta}\def\fs{\footnotesize}
\def\g{\gamma}
\def\G{\Gamma}\def\l{\lambda}\def\L{\Lambda}\def\o{\omiga}\def\p{\psi}
\def\se{\subseteq}\def\seq{\subseteq}\def\Si{\Sigma}\def\si{\sigma}\def\vp{\varphi}\def\es{\varepsilon}
\def\sc{\scriptstyle}\def\ssc{\scriptscriptstyle}\def\dis{\displaystyle}
\def\cl{\centerline}\def\ll{\leftline}\def\rl{\rightline}\def\nl{\newline}
\def\ol{\overline}\def\ul{\underline}\def\wt{\widetilde}\def\wh{\widehat}
\def\rar{\rightarrow}\def\Rar{\Rightarrow}\def\lar{\leftarrow}
\def\Lar{\Leftarrow}\def\Rla{\rightleftarrow}\def\bs{\backslash}
\def\ra{\rangle}\def\la{\langle}\def\hs{\hspace*}\def\rb{\raisebox}
\def\ni{\noindent}\def\hi{\hangindent}\def\ha{\hangafter}
\def\vs{\vspace*}
\def\hom#1{\mbox{\rm Hom($#1,#1$)}}
\def\thebeg{\vskip8pt\par\ni}
\def\theend{\vskip5pt\par}
\def\chabeg{\pagebreak\par}
\def\chaend{\vskip20pt\par}
\def\secbeg{\vskip15pt\par}
\def\secend{\vskip10pt\par}
\def\exebeg{\vskip10pt}
\def\exeend{\vskip6pt}
\def\undot#1{\mbox{$\mbox{#1}\hs{-1.5ex}_{_{\dis\centerdot}}\,\,$}}
\def\qed{\hfill\mbox{$\Box$}}
\def\C{\mathbb{C}}
\def\Q{\mathbb{Q}}
\def\ii{\mbox{\,{\bf i}\,}}
\def\jj{\mbox{\,{\bf j}\,}}
\def\AA{{\cal A}}
\def\BB{{\cal B}}
\def\DD{{\cal D}}
\def\EE{{\mbox{\bf 1}}}
\def\OO{{\mbox{\bf 0}}}
\def\kk{{\mbox{\bf k}}}
\def\R{\mathbb{R}}
\def\F{\mathbb{F}{\ssc\,}}
%\def\similar{\rb{-4pt}{\mbox{\,\~\,}}}
\def\similar{\backsim}
\def\Llra{\Longleftrightarrow}
\def\Lra{\Longrightarrow}
\def\Lla{\Longleftarrow}
\def\mat#1#2{\mbox{$\left(\begin{array}{#1}#2\end{array}\right)$}}
\def\det#1#2{\mbox{$\left|\begin{array}{#1}#2\end{array}\right|$}}
\def\eset{\emptyset}
\parindent=5ex
\setlength{\parindent}{0pt}
\setlength{\parskip}{1ex plus 0.5ex minus 0.2ex}
\newtheorem{Example}{\text{例}}

\begin{document}
\begin{CJK*}{UTF8}{gbsn}

\date{}
\title{$\pi$的计算}
\author{李青林,5110309074\\郑辉煌,5110209289}
\maketitle

\section{实验任务1}
	
	从单位圆正方形开始,成倍增加边数,求$\pi$的近似值\\
	
	设边数为$4\cdot{}2^n$时,边长为$a_n$\\
	易得\\
	$$a_{n+1}=\sqrt{\left(\dfrac{a_n}{2}\right)^2+\left(1-\sqrt{1-\left(\dfrac{a_n}{2}\right)^2}\right)^2}=\sqrt{2-\sqrt{4-a_n^2}}$$\\
	$4\cdot{}2^n$边型面积为\\
	$$S_{n+1}=4\cdot{}2^n\cdot{\dfrac{1}{4}\cdot{a_n}}=2^na_n\approx\pi$$\\
	初始条件为$a_0=\sqrt{2}$
	
	MATLAB代码:
	\begin{lstlisting}[language = matlab]
function task1(n)
    a = zeros(1,100);
    a(1) = sqrt(2);
    for i = 2:n
        a(i) = sqrt(2-sqrt(4-a(i-1)^2));
    end
    pi = 2^n * a(n);
    vpa(pi,20)
end
	\end{lstlisting}
	$n$取不同值时,$\pi$的值\\
	\begin{tabular}{|c|c|}
	\hline
	$n$ & $\pi$\\
	\hline 
	10 & 3.1415914215046352176 \\ 
	\hline 
	14 & 3.1415926548075892022 \\ 
	\hline 
	20 & 3.1415965537048196858
	 \\ 
	\hline 
	\end{tabular} 
	
	可以看出,使用该方法得到的结果精度比较低,并且由于计算过程中的误差积累,反而导致迭代次数增加,误差反而增加的事情发生
	
\section{实验任务2}
\subsection{}
	用反正切函数的幂级数展开式结合有关公式求$\pi$,若要精确到50位数字,试比较简单公式和Machin公式所用的项数.\\
	$$\mathrm{arctan}x=\sum_{n=1}^{\infty}(-1)^{n-1}\dfrac{x^{2n-1}}{2n-1}$$
	\textbf{简单公式}\\
	$$\dfrac{\pi}{4}=\mathrm{arctan}\dfrac{1}{2}+\mathrm{arctan}\dfrac{1}{3}$$
	展开,变形得\\
	$$\pi=4\sum_{n=1}^\infty (-1)^{n-1}\dfrac{1}{2n-1}\left(\dfrac{1}{2^{2n-1}}+\dfrac{1}{3^{2n-1}}\right)$$
	MATLAB代码:
	\begin{lstlisting}[language = matlab]
function task2_1(n)
    ret = 0; k = 1;
    for i=1:n
        ret = ret + k / (2 * i - 1) * ( 1 / 2^(2 * i - 1) ...
        + 1 / 3^(2 * i - 1));
        k = k * -1; 
    end
    vpa(ret*4,50)
end
	\end{lstlisting}
	\begin{tabular}{|c|c|}
	\hline
	$n$ & $\pi$\\
	\hline 
	19 & 3.1415926535899427740616829396458342671394348144531 \\ 
	\hline 
	20 & 3.1415926535897932384626433832795028841971693993751 \\ 
	\hline 
	\end{tabular} 
	
	\textbf{Machin公式}\\
	$$\dfrac{\pi}{4}=4\mathrm{arctan}\dfrac{1}{5}-\mathrm{arctan}\dfrac{1}{239}$$
	展开,变形得\\
	$$\pi=4\sum_{n=1}^\infty (-1)^{n-1}\dfrac{1}{2n-1}\left(\dfrac{4}{5^{2n-1}}-\dfrac{1}{239^{2n-1}}\right)$$
	MATLAB代码:
	\begin{lstlisting}[language = matlab]
function task2_2(n)
    ret = 0; k = 1;
    for i=1:n
        ret = ret + k / (2 * i - 1) * ( 4 / 5^(2 * i - 1) ...
        - 1 / 239^(2 * i - 1));
        k = k * -1; 
    end
    vpa(ret*4,50)
end
	\end{lstlisting}
	
	\begin{tabular}{|c|c|}
	\hline
	$n$ & $\pi$ \\
	\hline 
	8 & 3.1415926535886029569155653007328510284423828125000 \\ 
	\hline 
	9 & 3.1415926535897932384626433832795028841971693993751 \\ 
	\hline 
	\end{tabular} 

	可以看出,要精确到50位,简单公式需要20项,而Marchin公式只需要9项,因此Marchin公式更精确
	
\subsection{}
	 验证公式\\
	 $$\dfrac{\pi}{4}=\mathrm{arctan}\dfrac{1}{2}+\mathrm{arctan}\dfrac{1}{5}+\mathrm{arctan}\dfrac{1}{8}$$
	 试试用此公式右端作幂级数展开完成任务2.1所需的项数
	 
	\begin{align*}
	&\mathrm{tan}\left(\mathrm{arctan}\dfrac{1}{2}+\mathrm{arctan}\dfrac{1}{5}+\mathrm{arctan}\dfrac{1}{8}\right)\\
	=&\dfrac{\dfrac{1}{2}+\mathrm{tan}\left(\mathrm{arctan}\dfrac{1}{5}+\mathrm{arctan}\dfrac{1}{8}\right)}{1-\dfrac{1}{2}\cdot\mathrm{tan}\left(\mathrm{arctan}\dfrac{1}{5}+\mathrm{arctan}\dfrac{1}{8}\right)}\\
	=&\dfrac{\dfrac{1}{2}+\dfrac{1}{3}}{1-\dfrac{1}{2}\cdot\dfrac{1}{3}}\\
	=&1=\mathrm{tan}\dfrac{\pi}{4}\\
	\end{align*}
	$\therefore$公式成立\\
	对公式展开,变形得\\
	$$\pi=4\sum_{n=1}^\infty (-1)^{n-1}\dfrac{1}{2n-1}\left(\dfrac{1}{2^{2n-1}}+\dfrac{1}{5^{2n-1}}+\dfrac{1}{8^{2n-1}}\right)$$
	MATLAB代码:
	\begin{lstlisting}[language=matlab]
function task2_3(n)
    ret = 0; k = 1;
    for i=1:n
        ret = ret + k / (2 * i - 1) * ( 1 / 2^(2 * i - 1) ...
        + 1 / 5^(2 * i - 1)+ 1 / 8^(2 * i - 1));
        k = k * -1; 
    end
    vpa(ret*4,50)
end
	\end{lstlisting}	
	\begin{tabular}{|c|c|}
	\hline
	$n$ & $\pi$\\
	\hline 
	19 & 3.1415926535899427740616829396458342671394348144531 \\ 
	\hline 
	20 & 3.1415926535897932384626433832795028841971693993751 \\ 
	\hline 
	\end{tabular} 
	
	要精确到50位,该公式需要20项,和简单公式相同。原因是$$\mathrm{arctan}\dfrac{1}{5}+\mathrm{arctan}\dfrac{1}{8}=\mathrm{arctan}\dfrac{1}{3}$$
\section{实验任务3}
	用数值积分计算$\pi$,分别用梯形法和Simpson法精确到10位数字,用Simpson法精确到15位数字.\\
	
	$$4\int_0^1\dfrac{1}{1+x^2}\mathrm{d}x=\pi$$
	梯形公式MATLAB代码:
	\begin{lstlisting}[language=matlab]
function task3_1(n)
    x=0:1/n:1;
    y=1./(1+x.^2);
    s=2*sum(y)-1-0.5;
    2*s/n
end
	\end{lstlisting}
	$n=100000$,结果为$3.141592653589588$\\
	即要精确到10位,需要划分$100000$个区间\\
	Simpson公式MATLAB代码:
	\begin{lstlisting}[language=matlab]
function task3_2(n)
    x = 0:1/n:1; m = length(x); s = 0;
    for i = 1:m-1
        s = s + (x(i+1)-x(i)) / 6  * (1 / ( 1 + x(i)^2) ...
            + 4 / (1 +(x(i) + x(i+1))^2 / 4) +... 
            1 / ( 1 + x(i+1)^2) );
    end
    vpa(s*4,15)
end
	\end{lstlisting}
	$n=18$,结果为$3.14159265357156$\\
	$n=50$,结果为$3.14159265358979$\\
	即Simpson公式只需要划分$18$个区间即可精确到10位,划分$50$个区间可以精确到$15$位\\
	因此Simpson公式比梯形公式更精确。
\section{实验任务4}
	\subsection{}
	用Monte Carlo法计算$\pi$,除了加大随机数,在随机数一定时可重复算若干次后求平均值,看能否求得5位精确数字?\\
	
	\textbf{Monte Carlo方法:}\\
	
	MATLAB代码(每次撒$k$个点,运行$n$次求平均):
	\begin{lstlisting}[language=matlab]
function task4_1(k, n) %rand k points, n test average
    sum = 0; a = zeros(n);
    for i = 1:n
        a(i) = monteCalPi( k );
        sum = sum + a(i);
    end
    vpa(sum / n, 10)
end
function y = monteCalPi( k )
    m = 0;
    for n = 1:k
        if rand(1)^2 + rand(1)^2 <= 1
            m = m + 1;
        end;
    end
    y = 4 * m / k;
end
	\end{lstlisting}
	取$k=10000,n=10000$时,算出$\pi=3.14157176$\\
	
	\textbf{模拟Buffon实验:}
	
	模拟方法:将针看作长为$l$的线段,随机确定中点到平行线的距离$x$和针与平行线的夹角$\alpha$,再判断相交\\
	MATLAB代码(平行线间距$d$, 针长$l(l\leq d)$, $m$次投针):
	\newpage
	\begin{lstlisting}[language=matlab]
function task4_2( l, d, m ) 
    n = 0;
    for k = 1:m
        x = unifrnd(0, d/2);
        alpha = unifrnd(0, pi);
        if x < 0.5 * l * sin(alpha)
            n = n+1;
        end
    end
    y = 2 * l / d * m / n;
    vpa(y, 10)
end
	\end{lstlisting}
	
	取$l=10,d=20,m=10000$,得$\pi=3.143665514$	\\
	实验证明,随机算法通常难以保证精度,并且计算结果的误差时大时小\\

\section{实验任务5}
	$e$是一个重要的超越数\\
	$$e=\lim_{n\to\infty}\left(1+\dfrac{1}{n}\right)^n$$
	$$e=1+1+\dfrac{1}{2!}+\cdots+\dfrac{1}{n!}+\dfrac{e^\theta}{(n+1)!}$$
	还有\\
	$$e=\left(1+\dfrac{1}{n}+\dfrac{k}{n^2}\right)^n$$
	试用上述公式或其他方法近似计算$e$
	
	使用公式$\displaystyle e=\lim_{n\to\infty}\left(1+\dfrac{1}{n}\right)^n$\\
	MATLAB代码:
	\begin{lstlisting}[language=matlab]
function task5_1( n )   
    e = (1 + 1/n) ^ n;
    vpa(e, 50)
end
	\end{lstlisting}
	
	取$n=10000$时,得$e=2.718145927$,精确到小数点后3位.
	
	使用公式$e=1+1+\dfrac{1}{2!}+\cdots+\dfrac{1}{n!}+\dfrac{e^\theta}{(n+1)!}$\\
	MATLAB代码:
	\begin{lstlisting}[language=matlab]
function task5_2( n )
    e = sym(1);
    temp = 1;
    for i = 1:n
        temp = temp * i;
        e = e + sym(1 / temp);
    end
    vpa(e, 10)
end
	\end{lstlisting}
	取$n=10$,得$e=2.718281801$,可以精确到小数点后7位\\
	对比可知,级数法精度更高.	
\end{CJK*}
\end{document}

